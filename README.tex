\documentclass{article}
\usepackage[utf8]{inputenc}

\title{Project Leibnix}
\author{Project Leibniz}
\date{November 2019}

\begin{document}

\section{Introduction}
\subsection{What is Project Leibniz?}

Project Leibniz is a collection of real-world motivated maths problems aimed at
secondary students to illustrate the power of calculus.

Calculus has a reputation as being abstract, impenetrable and poorly
understood. Many secondary students struggle with this topic because it is not
taught well. The aim of this project is to allow students to improve their
calculus skills by practicing with real-world problems to inspire their
curiosity and help them retain the knowledge.


\section{Prerequisites}

The aim of this project is not to teach calculus from scratch, students should
have been exposed to the basic ideas including

\begin{itemize}
	\item The power law of derivatives
	\item The chain rule
\end{itemize}

\section{Structure}

This project is divided into two main sections: differential and integral calculus. Within each section there are 4 levels which are defined as:

\begin{enumerate}
	\item Level one: The difficulty is approximately aimed at people who
	are very new to calculus. Problems only involve two variables i.e. IGCSE
	extended maths or equivalent
	\item Level two: The difficulty is approximately designed to challenge
	gifted 16 year olds i.e. IGCSE additional maths or equivalent
	\item Level three: The difficulty is aimed at A-level, IB maths
	students or equivalent
	\item Level four: This is for people looking to sharpen their skills
	prior to begining university
\end{enumerate}

\section{Contribute}

There are several ways you can contribute:

If you are a student:
\begin{itemize}
	\item * we would love feedback on the appropriacy and difficulty of the problems
\end{itemize}

Graduates:
\begin{itemize}
	\item attempt the problems to find any errors that may have slipped in
	\item contribute interesting new problems
\end{itemize}

\end{document}
